\documentclass[draftclsnofoot, onecolumn, 10pt]{IEEEtran}

\title{\huge Collaborative Threat Mitigation: \\Social Devices Working to Mitigate Security Threats}

\author{Oregon State University\\CS 461\\Fall 2017-2018\\\\Prepared By:\\Kyle Prouty}

\usepackage{titling}

\usepackage[margin=0.75in]{geometry}

\usepackage{graphicx}

\parindent=0.0in

\parskip=0.35in

\begin{document}
\begin{titlingpage}
    \maketitle 
    \begin{abstract}
		\noindent 
Mitigating security threats is essential in today's world of interconnected devices. This project aims to enable loosely coupled cyber physical systems to collaborate together in order to mitigate security threats on a system of socially collective nodes. For this system to mitigate threats introduced into the system, individual nodes will coordinate together, without any rigid logical configurations, in order to self organize to accomplish their shared objective. Individual nodes will have no knowledge of the system until they are added to the system and then will socially collaborate with the other nodes in the system to determine the role they will fill to coordinate with the group. This system will not be identifying the threats themselves, but threats will be introduced into this system. Once a threat has infected the system, each node will work together with the other nodes to figure out how serious of a threat has been introduced. After the threat level has been determined, the group of nodes will then collaborate together to mitigate the threat. When a threat has been determined by the leader to have been successfully mitigated, the nodes will then self organize again to share all previous infected resources.
    
    \end{abstract}
\end{titlingpage}

\section*{Problem Definition}
The world around us is filling with more and more interconnected devices that all have the ability to talk to one another but currently don't. Our system will solve the problem of mitigating security threats by allowing these interconnected devices to collaborate together in order to abate any threats introduced into this system. Nodes on this system will not have any rigid logical configurations or be able to talk to any centralized command center. They will also not be allowed to talk to the outside world, only other nodes in this system. These nodes will have to gather information from the other nodes about how they need to incorporate themselves into the system. Since no nodes will have prior knowledge about the other nodes or the current state of the system. All questions that a node has to answered will be answered at the current state the question was asked, meaning that a node only asks his partners for info when it is time to make a decision. This will reduced consumption and always allow the nodes to be using the most current information when accomplishing goals. All nodes will have to be smart enough to make decisions in conjunction with the other nodes in order to mitigate the current threat on the system. Nodes will have to be able to quickly adjust themselves and reorganize with the group in order to collaborate successfully. Any node can be disconnected from the system at any time and likewise any node can be added to the system at any time but there must always be at least one node on the system. While nodes are coordinating to mitigate the current threat on the system, another threat can be introduced which then forces the nodes to halt, reorganize, and then continue mitigating all threats that have been introduced into the system. 


\vspace{0.0 in}

\section*{Problem Solution}
To solve this problem, a network of nodes will be created. These nodes will act as individuals but will socially collaborate together with other nodes as a group to mitigate incoming security threats. Individual nodes can have their services shutdown one by one or even removed from the group, when this happens the group can self organize in order to continue with their overall objective. Nodes entering into the group will have no prior knowledge of anything, when a threat is detected the nodes will then collect the information they need at that moment in time to work together to solve the problem. This group of nodes will always decide on a leader, which can be re-decided with every new node that is introduced. A leader will be important since each node has the potential to be introduced with a threat. When a threat is introduced, I node will inform the leader of his infection. Then the leader will determine the level of threat to the overall system. When the threat level has been determined, all nodes will socially collaborate to mitigate the threat on the infected node. Once a threat has been mitigated, any node still remaining will then reorganize into their new roles to fill in for any node that is no longer able to fill their role. Once the threat has been mitigated fully, the infected node will be cleared by the current leader who will then tell the group to self organize again to utilize the previously infected resources. 



\section*{Performance Metrics}
There are many different ways that a threat could be mitigated from our system. A threat level can be assigned to an incoming infection. Each threat level will correspond to a different objective that the collaborating nodes will have to complete in order to mitigate the incoming threat. At a lower threat level a node could be blocked from having access to the infected resources, allowing other nodes to grab those resources to fill in the gap. After a threat has been determined to have been mitigated, the blocked resources on the previously infected node could then be freed, allowing that node to coordinate in the future using those previously blocked resources. If a high level threat is detected, then all the nodes could coordinate to find out which connected power plug is being utilized by that printer, and then shut that entire node down, removing all of its resources from the group. Our team will implement a visual system to show a threat moving through the system. This visual aid will demonstrate how the threat was introduced, who it infected, how the nodes will able to inform themselves of their infection, and what the leader determines the threat level to be. This visual will then show how the nodes collaborate to mitigate this threat. In the end the ultimate metric to confirm our solution to the problem will be if nodes on our system can self organize to work together to accomplish a common goal. 


\vspace{2 in}

\noindent\begin{tabular}{ll}
\makebox[2.5in]{\hrulefill} & \makebox[2.5in]{\hrulefill}\\
Name & Date\\[8ex]%
\makebox[2.5in]{\hrulefill} & \makebox[2.5in]{\hrulefill}\\
Name & Date\\[8ex]%
\makebox[2.5in]{\hrulefill} & \makebox[2.5in]{\hrulefill}\\
Name & Date\\[8ex]%
\makebox[2.5in]{\hrulefill} & \makebox[2.5in]{\hrulefill}\\
Name & Date\\[8ex]%
\end{tabular}

\end{document}